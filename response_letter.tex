\documentclass[11pt, letterpaper]{article}
\usepackage[top=1in, bottom=1in, left=1in, right=1in]{geometry}

% Math, graphics, and bibliography
\usepackage{amsmath,amssymb,amsfonts,mathrsfs,mathtools}

% My preferred fonts
\usepackage[T1]{fontenc}
\usepackage{lmodern}
\usepackage[sc]{mathpazo}
\usepackage{textcomp}

\usepackage{color}
\usepackage{tcolorbox}
\tcbuselibrary{breakable}
\tcbuselibrary{skins}
\usepackage{units}

\usepackage[parfill]{parskip} % For block paragraphs

\definecolor{gray}{gray}{0.4}
\newenvironment{reviewer}{\itshape\color{gray}}{}

\tcbset{skin=enhanced}

\newenvironment{manuscript}[1]{\begin{center}\begin{tcolorbox}[colback=green!5!white,colframe=green!75!black,width=0.8\textwidth,title={#1},breakable,fonttitle=\bfseries]}{\end{tcolorbox}\end{center}}


\begin{document}

\begin{flushright}
\today\\[2ex]
{\itshape Francis J. Doyle III}\\
Dept.\ of Chemical Engineering\\
Univ.\ of California, Santa Barbara\\
Santa Barbara, CA 93106-5080\\
\end{flushright}

{\itshape Christopher V. Rao}\\
Associate Editor\\
PLOS Computational Biology\\

{\itshape Douglas Lauffenburger}\\
Associate Editor\\
PLOS Computational Biology\\

Thank you for your email on April 20th, 2015, inviting us to submit a revised version of our manuscript, ``Quantifying stochastic noise in cultured circadian reporter cells'' (MS: PCOMPBIOL-D-15-00431) by Peter C. St. John and Francis J. Doyle~III.

We thank the reviewers for their detailed comments of the manuscript and suggestions for how to improve the work.
We believe addressing these concerns and suggestions have greatly improved the manuscript.

Please see our detailed responses to the reviewers below: 

\section*{Reviewer \#1}
\begin{reviewer}
The premise of the manuscript is quite interesting and pertinent to the study of the circadian clock and of biological oscillators more generally. 
Under the assumptions that the individual oscillators are self-autonomous with non-damping amplitudes and do not exert any influence on each other (which can be reasonable in certain types of dissociated cell cultures), the population-level damping rate is related to the level of stochastic noise in the individual oscillators. 
Dose-dependent effects of small molecule modulators on the noise in the clock can then be studied via the damping rate of the population, which is much easier than trying to record individual oscillators to measure such effects. 
The authors' further analysis leads them to the intriguing conjecture that the circadian clock has achieved an optimal balance between amplitude and stochastic noise.

Overall the approach is appealing and seems promising, with plausible future applications. However, some issues require further attention to ensure the method is effective and sound.

1) What exactly is meant in the manuscript by ``stochastic noise'' is not entirely clear. It would be helpful if the authors explained what is being covered under this term: Extrinsic noise? Intrinsic noise? Only noise generated within each cell by the molecular mechanism, or larger scale effects as well? What specifically are the sources of the noise under consideration, and what are the consequences for the individual oscillators' amplitude, phase, and period (and variability thereof)?
\end{reviewer}

We thank the reviewer for their overall positive reception of the manuscript. 
In regards to their first critique, we mainly are considering intrinsic noise generated by the molecular mechanism.
While sources of extrinsic noise are certainly present, experimental studies (i.e., Herzog {\itshape et. al., JBR} 2004) have largely concluded that period lengths of circadian oscillations are more variable at the single-cell level (i.e., peak-to-peak time) than at the population level (differences in mean period between cells).
We agree that additional discussion of this point in manuscript is necessary, and have therefore expanded the introduction to include an explicit description of the different sources and effects of stochastic noise:

\begin{manuscript}{Page 3}
 Bioluminescence rhythms at the cell culture or tissue-level are the result of the collective behavior of thousands of cells.
 {\bfseries
At the single-cell level, transcription is strongly affected by cellular noise, comprised of both {\itshape intrinsic} and {\itshape extrinsic} sources.
Intrinsic noise is generated from the low molecular counts of the mRNA and protein species involved.
As a result, bioluminescence traces of individual cells are stochastic, with significant variability in both amplitude and period length (14). 
Extrinsic noise results from heterogeneity between cells, such as differences in size or physical environment, leading to differences on a cell-to-cell basis in expected period and amplitude.
For circadian systems, intrinsic noise has been shown to play a larger role: a single cell's variability in period from cycle-to-cycle has been shown to be larger than the variability in mean period length between cells (2).
}
In cell cultures which lack cell-to-cell coupling, it has been shown that  stochastic noise at the single-cell level is manifested in damped oscillations at the population-level as individual oscillators gradually drift out of phase (14, 15).  
\end{manuscript}


\begin{reviewer}
2) Damping of the population level oscillation is typically caused by two main factors: heterogeneity in period among the oscillators and stochastic variation in cycle length from cycle to cycle in each oscillator due to noise. 
The authors should address how these two factors contribute to the damping rate and how to distinguish them. 
For instance, what if a clock modulator is causing increased variability in period across the population, rather than increased stochastic noise? That is, the period of certain oscillators could be changed more than that of other oscillators (which could occur given that the system is highly nonlinear); you could even have the average period remain stable but some oscillators have their periods shortened by the treatment while others are lengthened. 
This would increase the damping rate, but not be related to a change in stochastic noise. 
A set of noise-free oscillators with heterogeneous periods will also damp at the population level, and damp faster if the variance in the period is increased.
\end{reviewer}

This is true, and it is likely not possible to effectively differentiate between the two sources - added supplemental figure.
\begin{itemize}
  \item We have demonstrated single-cell noise is {\itshape sufficient} to explain the observed changes in population damping. (sorting results, mathematical model predictions)
  \item We (and others previously) demonstrate single-cell variability is greater than cell heterogeneity.
\end{itemize}
Regardless, it is possible damping rates are changed through differences in extrinsic noise. The insight we generate is still valid, however, since the dynamic consequences (faster desynchronization of cells) are similar.

\begin{reviewer}
The authors do have a paragraph addressing the effect of heterogeneity, but they seem to assume the heterogeneity will remain unchanged by the clock modulators. However, looking at Table 1, this assumption doesn't hold. The variance is enormously increased by the perturbation, and one would expect this increased variability is likely occurring at the single cell level as well.
\end{reviewer}

\begin{itemize}
  \item Not necessarily. Table 1 shows that the variance in population-level periods is increased as a result of the {\itshape entire} siRNA screen.
  \item At the single-cell level, each cell in a particular well would experience a perturbation from the same siRNA, which we would expect to have the same effect across cells.
\end{itemize}
I'll perhaps re-word the figure caption to make this point more clear.

\begin{reviewer}
3) The claim that the stochastic noise can be altered independently from amplitude doesn't seem sufficiently well substantiated. As the authors point out, a major source of noise in single cell rhythms is the low molecular counts of clock mRNAs and proteins. Increasing the amplitude generally implies larger molecular counts and therefore reduced noise. Also, as shown in Figure 5, siRNA perturbation led to greatly increased variability in period and amplitude, which could be the main cause of the increased damping rate (rather than increased stochastic noise). That is, the effect of the perturbation could be to amplify the heterogeneity, rather than the individual noise levels. For instance, a change in a key rate parameter governing the nonlinear molecular clock mechanism might lead not only to a change in the mean period across the population, but also to a change in the spread of the period distribution. 
\end{reviewer}

Again, I'll point out each siRNA has a distinct effect on the clock. Taken in aggregate, these distinct effects result in increased variance at the well-level.
\begin{itemize}
  \item {\itshape Increasing the amplitude generally implies larger molecular counts and therefore reduced noise} - This is not necessarily the case. Noise can come from many locations within the gene regulatory network which are not represented by the reporter. In general, noise and amplitude are not correlated (maybe there's data for this?)
\end{itemize}


\begin{reviewer}
4) Using the PER2::LUC fibroblast recordings to test the ideas provides a clever test of the method. However, there are still some concerns about whether the method would work as proposed. For instance, the assumption is that the cells will have synchronized phases at the beginning of the experiment. For the test, the authors aligned the individual cell traces, but in an experiment the cells would presumably be synchronized using a medium change or some other synchronizing pulse. However, such perturbations also tend to transiently increase the amplitude of the individual cells. How would that effect be distinguished from the damping due to stochastic noise and phases drifting apart?
\end{reviewer}

The windowing function would likely not resemble a damped sinusoid. I'll refer to the previous work. For the siRNA screen, the synchronizing pulse (medium change) is consistent across all the wells, and would therefore likely not affect our overall analysis.

\begin{reviewer}
5) In the methods, what is meant by the (8hr,258hr) wavelet component? What type of wavelet transform/filter is being applied, and what do the 8hr and 258hr values refer to? Similarly, please clarify what the (1hr,8hr) wavelet component is.
\end{reviewer}

I'll go into more detail in the manuscript regarding the discrete wavelet transform.

\section*{Reviewer \#2}
\begin{reviewer}
I have enjoyed reading the paper by St.\~John and Doyle - its an innovative analysis of single cell circadian data and I think it will be of interest to a wide audience. 
In particular the finding that dephasing occurs mostly due to noise rather than period heterogeneity is fascinating. 
My major concern is the statement that ``Higher noise results in faster damping in population-level rhythms'' - though I agree that this follows from their data analysis, I am not so convinced that this can generally "serve as an accurate measure of cell-autonomous stochastic noise", as mentioned in the abstract and other parts of the paper. 
For example see Fig. 9 of the simulation study presented in Thomas, Philipp, Hannes Matuschek, and Ramon Grima. 
``Intrinsic noise analyzer: a software package for the exploration of stochastic biochemical kinetics using the system size expansion.'' PloS one 7.6 (2012): e38518.
Therein it is shown, using a model of a rudimental circadian oscillator involving strong negative feedback, that the population-level rhythm predicted using ensemble averaged stochastic simulations (of uncoupled cells) shows damped oscillations which the rate equations (deterministic and hence no noise) miss!
That is in this case noise induces synchronous oscillations at the population level, rather than damping them! This is, i believe, an example which runs counter to the author's claims about noise leading to damped population level oscillations, generally. 
I would suggest the authors to discuss this possibility in the Discussion, for completeness sake, and also as a cautionary note on the use of damping as a measure of single cell noise.
\end{reviewer}

Here I'll mainly point out that {\itshape damping} does not imply reduced amplitude. Damping is the rate at which the amplitude decays.
Certainly for a noise-induced oscillator, the presence of noise will result in higher amplitudes. Our comments apply mainly to changes in noise and the resulting changes in exponential amplitude decay.
I'll add a sentence or two explaining the difference between amplitude and damping more clearly.

\section*{Reviewer \#3}

\begin{reviewer}
These authors expand on previous work (in Biophys J 2014) where they showed that random phase diffusion in oscillators cause population level signals to dampen exponentially. They show that this property can be exploited to infer changes in noise levels in the circadian oscillators of single cells from population level measurements. This is a potentially very useful method, and I would be looking forward to try it on our own data. I have concerns as outlined below, but I want to stress that I am generally favorable to this work and I tried to keep the suggested revisions and additional computations requested reasonable, none of them should be too time-consuming in relation to the added value.
\end{reviewer}

great!

\begin{reviewer}
Main concerns:

M1. The author's comments notwithstanding, there are some concerns regarding the possible contribution of period heterogeneity to the quantified population damping. Specifically, the decays in Fig 3A actually looks somewhat like the decay in theoretical Figure S6, right panel, which was computed assuming period heterogeneity. Of course, the authors convincingly show that population period heterogeneity is less than peak-to-peak variation in percent. I still request a control experiment, in which simulated time-series of observed peak-to-peak variation AND observed population period heterogeneity are analyzed with the authors' algorithm. The method should be shown to work as for instance noise levels are increased or decreased also for this case. This could be a complement to Figure 1C, where the same thing was done for oscillators with identical period.
\end{reviewer}

Similar to Reviewer \#1's concerns. This is largely an identifiability issue, as the particular sources of noise cannot be confidently identified from the population-level data (only overall noise).
We have added the control experiment demonstrating this.

\begin{reviewer}
M2. The method is of potential use for many labs including my own. What I was missing, however, are practical guidelines: How many samples are needed, which sampling rates, etc. are required for the method to work and yield ? How does it scale with increasing or decreasing sampling rates? Importantly, how do I decide whether a possible change in damping and thus noise level is significant, can one bootstrap this? Should one always use the fitting procedure with matrix pencil method, Hilbert transform to align phases, etc.? Here, a few trials on simulated time-series may suffice to provide such information.
\end{reviewer}

I'll add a paragraph on practical considerations for implementing the algorithm. Sampling rate is less important than overall recording duration. There are many ways to do the nonlinear regression and analysis, which likely depends on the data being analyzed. For looking specifically at the damping rate, the best method is likely detrending, followed by a semilog plot of window vs.\ time, from which more traditional statistical techniques on comparing two regressions (ancova, bayesian inference)


\begin{reviewer}
Minor concerns:

Relating to question M1, the authors refer to their previous work, ref. 23, where it was shown that phase drift leads to an exponentially decreasing population signal. Otoh, period heterogeneity would lead to a time-squared exponent; where is the latter formula derived? If this was done by Rougemont and Naef, they should be cited.
\end{reviewer}

An additional reference for Rougemont and Naef can be added, as they do discuss a variant of this. However, the idea that ballistic particles diffuse according to $t^2$ (and brownian particles according to $t$) is very old.

\begin{reviewer}
On p. 3, comments regarding reference 17: these authors rather submitted that either damped or self-sustained oscillators describe the data equally well, which actually more justifies the present method, which assumes self-sustained oscillations.
\end{reviewer}

This point will be clarified (our main idea is that noise is important), although our method does not assume self-sustained oscillations in the deterministic limit.

\begin{reviewer}
The method concerns primarily detection of noise levels, in the introduction it is emphasized that an amplitude boosting may ultimately be desirable in e.g. a clinical setting. Introduction would be strengthened by also including more use cases for the detection of noise strength.
\end{reviewer}

Will be added.

\begin{reviewer}
For the discussion: authors argue that the amplitude/noise ratio seem to have evolved to something close to an optimum. Yet, in the paper by Liu et al. in Cell 129, 2007, Cry2 knockouts seemed to have even stronger rhythms?
\end{reviewer}

We don't have access to the raw data of Liu et. al.,, but higher amplitude (as suggested by Liu) is not the same as stronger rhythms. I will check our database (many Cry2 siRNA replicates) to see if this trend is replicated in the siRNA screen.


\section*{Reviewer \#4}

\begin{reviewer}
Circadian rhythms play important roles in maintaining biological homeostasis. Recent research suggests that environmental conditions, like night work or an unusual feeding schedule, may have important implications for developing metabolic disease. Therefore it is of interest to identify therapeutic means to manipulate the circadian rhythms. Here, St. John and Doyle explore the relationship between stochastic noise and circadian oscillations, as typically observed at the cell population level and suggest a method to infer stochastic noise properties from population behavior. Overall, the work seems too tightly focused on a narrow scientific niche and the method for deducing noise properties is not convincing. The major criticisms are as follows:

1. The relationship between stochastic noise or cell-to-cell variability and how that impacts a population response is not particularly new. The observation that cell to cell variability gives rise to a dampened response at the cell population level has been observed before for other signaling systems, such as NF-kappaB (e.g., Nelson et al. Science (2004) 306:704-708). It is then unclear as to what really is new here.
\end{reviewer}

What is new is that this phenomenon can be quantified to give additional information on stochastic noise. From the datasets, this is an independent quantity (from period, amplitude), and therefore likely provides new insight into the underlying oscillator.

Also, we're attempting to downplaying the effect of cell-to-cell variability.

\begin{reviewer}
2. The authors use a leap of logic to justify their approach in extracting stochastic noise parameters from cell population data. First the authors show that by fitting a collection of models to single cell data and then averaging these mathematical models together, they can predict that the cell population response will become dampened when there is significant variability in the single cell responses (inductive reasoning?). Based on this demonstration, they then develop an approach that does the opposite: from population data, they try to deduce the values for the stochastic noise properties (deductive reasoning?). The problem seems to be that there could be other explanations for why miRNAs within a library affect a signaling network, such as miRNAs limit the efficiency of the biochemical signaling network that underpins the oscillatory behavior. Overall, the approach is not convincing in providing the claimed insight. 
\end{reviewer}

We are demonstrating that single-cell noise is {\itshape sufficient} to explain the observed damping behavior, and indeed quantitatively accurate predictions for population-level damping rate are obtained from a simple model not fit to the noise properties.

Indeed, it would be great if single-cell data was available for each siRNA perturbation. This is likely not possible, as single-cell data is difficult to collect.

Cells are likely not arrhythmic as a result of siRNA perturbation, as the siRNA method has been well-adopted by the circadian community for effects on amplitude and period.

\begin{reviewer}
Minor comments:
1. ``which'' is used incorrectly pretty much throughout the manuscript. should be ``that''
\end{reviewer}

This has been corrected.


We would like to once again thank all our reviewers, as we feel our manuscript has benefited greatly from their suggestions and critiques.

\end{document}
