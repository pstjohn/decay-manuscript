\documentclass[11pt, letterpaper]{article}
\usepackage[top=1in, bottom=1in, left=1in, right=1in]{geometry}

% Math, graphics, and bibliography
\usepackage{amsmath,amssymb,amsfonts,mathrsfs,mathtools}

% My preferred fonts
\usepackage[T1]{fontenc}
\usepackage{lmodern}
\usepackage[sc]{mathpazo}
\usepackage{textcomp}

\usepackage{color}
\usepackage{tcolorbox}
\tcbuselibrary{breakable}
\tcbuselibrary{skins}
\usepackage{units}

\usepackage[parfill]{parskip} % For block paragraphs

\definecolor{gray}{gray}{0.4}
\newenvironment{reviewer}{\itshape\color{gray}}{}

\tcbset{skin=enhanced}

\newenvironment{manuscript}[1]{\begin{center}\begin{tcolorbox}[colback=green!5!white,colframe=green!75!black,width=\textwidth,title={#1},breakable,fonttitle=\bfseries]}{\end{tcolorbox}\end{center}}
% For hyperref links

\usepackage[linktoc=all]{hyperref}
\hypersetup{
    colorlinks,
    citecolor=black,
    filecolor=black,
    linkcolor=black,
    urlcolor=black
}

\usepackage[margin=1cm]{caption}
\renewcommand{\thefigure}{R\arabic{figure}}%

\begin{document}

\begin{flushright}
\today\\[2ex]
{\itshape Francis J. Doyle III}\\
Dept.\ of Chemical Engineering\\
Univ.\ of California, Santa Barbara\\
Santa Barbara, CA 93106-5080\\
\end{flushright}

{\itshape Christopher V. Rao}\\
Associate Editor\\
PLOS Computational Biology\\

{\itshape Douglas Lauffenburger}\\
Deputy Editor\\
PLOS Computational Biology\\

Thank you for your email on June 23th, 2015, inviting us to submit a further revised version of our manuscript, ``Quantifying stochastic noise in cultured circadian reporter cells'' (MS: PCOMPBIOL-D-15-00431) by Peter C. St. John and Francis J. Doyle~III.

We are pleased to hear that two of the four reviewers recommend our work for publication in {\itshape PLoS Computational Biology}. We thank the reviewers for their second round of thoughtful critiques, which we feel further improves the quality of the manuscript.

Please see our detailed responses to the reviewers below: 

\section*{Reviewer \#1}
\begin{reviewer}
The revision improved the manuscript, but a couple of minor issues remain to be addressed.

The added discussion on types of noise should be revised to include the work of Elowitz et al (Science 2002) on extrinsic vs intrinsic noise. The Elowitz study does quantitatively distinguish extrinsic vs intrinsic noise in their experiments, while the revised manuscript states this is only possible in theory (so the wording should be altered at the bottom of page 6 and the Elowitz study cited). 
\end{reviewer}

We thank the reviewer for recommending this important study, and agree that further discussion of existing work on biological noise is useful in the introduction. We have therefore added an additional point to the paragraph on intrinsic and extrinsic noise sources:

\begin{manuscript}{Page 3}
Extrinsic noise results from heterogeneity between cells, such as differences in size or physical environment, leading to differences on a cell-to-cell basis in expected period and amplitude. 
{\bfseries The effects of noise in biological systems has been well-studied, and relative amounts of intrinsic and extrinsic noise can be identified through carefully designed single-cell experiments (15). }
For circadian systems, intrinsic noise has been {\bfseries suspected} to play a larger role: a single cell's variability in period from cycle-to-cycle is larger than the variability in mean period length between cells (2).
\end{manuscript}

For the discussion on page 6 where we describe what can be inferred from experimental data, we had restricted our claims to population-level recordings - whereas the Elowitz study used data from single-cell imaging to differentiate between noise sources. We nevertheless acknowledge the potential for misunderstanding and have added further text describing experimental techniques which might differentiate between noise sources (with a further citation to the work done by Elowitz, {\itshape et al}):

\begin{manuscript}{Page 7}
While differentiating between intrinsic and extrinsic noise sources from population-level data is possible in theory (Figure S6), these differences are likely not identifiable from typical {\bfseries population-level} data (Figure S7). 
{\bfseries This distinction would likely be possible with single-cell level data, as has been done in other experimental systems (15). 
However, such an experiment would likely not be amenable to high- throughput methods.}
Differences in damping rates are therefore best viewed as representative of changes to overall stochastic noise from both intrinsic and extrinsic factors. 
Since both types of noise are important to determining the overall function of population-level rhythms, damping rates are still a valuable method of quantifying stochastic noise.
\end{manuscript}

\begin{reviewer}
Also, in the revised part of the introduction addressing the role of noise in characterizing oscillations, the work by Bratsun et al (PNAS 2005) should be mentioned as another caution. For instance, they point out that ``the 'noise signature' of a system dominated by delay can be similar to a system dominated by intrinsic noise. This result implies that care must taken in attributing variability to purely stochastic sources, because delay-induced variability can appear empirically similar to fluctuations arising from intrinsic noise.''
\end{reviewer}

In Bratsun {\itshape et al, PNAS} 2005, the authors demonstrate that variability in the amplitude of gene transcription can occur both through intrinsic noise as well as through fast, deterministic oscillations. While it might be possible that fast oscillations in gene transcription (in addition to the slower, 24-hour circadian rhythms) underly most of the variation seen in ultradian frequencies, it would then be unlikely that we would observe the random drift in phases. Since our study deals mainly with the phase precision rather than the variability in transcription, we have decided to omit this reference to avoid confusing the reader.

\begin{reviewer}
In the new expanded paragraph discussing the effect of cell period heterogeneity, the authors dismiss the significance of intrinsic differences in period across the population compared to cycle-to-cycle fluctuations and don't directly quantify the relative contributions. However, Cohen et al (Journal Theoretical Biology 2013) quantified it for this same set of fibroblasts using a Bayesian hierarchical model and found the standard deviation of periods across the population to be 0.89h compared to the within-cell variability 1.43h. So while cycle-to-cycle fluctuations are indeed larger, the variability across the population is still quite substantial, and these values should be mentioned in this section to quantify the relative contributions of these sources of variability to the overall decay rate.
\end{reviewer}

Indeed, we agree with the reviewer that adding quantitative measures adds to the discussion of the relative importance of intrinsic vs.\ extrinsic period variability. We have therefore added these values from both studies. In addition, we have softened some of the language to avoid dismissing the effect of extrinsic variability, which likely indeed plays a non-negligible role in determining the damping rate.

\begin{manuscript}{Page 6}
  While it is true that the dephasing of a group of oscillators can be caused by both variance in the mean period as well as cycle-to-cycle variability, intrinsic stochastic noise {\bf may} play a more significant role. 
{\bf We} show that there is greater variance in period on a cycle-to-cycle basis than on a cell-to-cell basis in cultured fibroblast cells {\bf (Figure S5): individual cell period lengths have an average inner quartile range (IQR) of 3.18 hrs, while cell-to-cell average period has an IQR of 1.55 h. 
These results are independently confirmed by a previous study using Bayesian modeling, which found a standard deviation of periods within cells of 1.43 h, and 0.89 h across the population (27). 
A similar} result has also been observed in dispersed SCN neurons, {\bfseries suggesting that while both intrinsic and extrinsic period heterogeneity likely contribute to the dephasing kinetics, cell-to-cell differences are} less severe than cell-autonomous noise (2).

{\bf It is also} possible that damping rate changes due to siRNA or small molecule perturbation could be manifested through altering the system's extrinsic noise.
\end{manuscript}

\section*{Reviewer \#4}

\begin{reviewer}
In short, the authors didn't really address the concern about applying their simple model for deductive reasoning. 
They claim that as the siRNA method is well-accepted by the circadian community that their interpretation should be legit. 
Biological response is proportional to dose and target. 
It seems implicitly that the authors are assuming that the DOSE of siRNA transfected into every individual cell is exactly the same. 
The biological response is then interpreted solely as being attributed to the target. 
One look at a flow cytometry result of a siRNA experiment will tell you that the dose is not the same between cells. 
Different siRNA also don't work equally well either. 
So again the criticism that the approach is not convincing in providing the claimed insight still is unaddressed.
\end{reviewer}

We regret that our previous revision did not address the reviewer's concern about the reasoning generated from each of our {\itshape in silico} experiments.
We have therefore added an expanded section of the Discussion in which we clearly lay out the insights we claim from each portion of the study:

\begin{manuscript}{Page 7}
  {\bfseries In this study we have shown that the damping rate of population-level circadian oscillations can be changed by genetic or pharmacological perturbations.
As populations-level rhythms are determined by the coherence of many individual cells, desynchronization due to stochastic noise is a likely explanation for population-level damping.
Using single-cell data, we showed that population-level damping rate is proportional to single-cell noise.
Furthermore, we used a computational model to predict the changes in damping rate from two small molecules, demonstrating that changes to intrinsic stochastic noise are sufficient to explain the observed damping rate changes.}
\end{manuscript}

In regards to the reviewer's specific comments on siRNA, we agree that the effect of an siRNA on each individual cell will indeed be varied.
This effect would be manifested in changing the extrinsic noise of the system, as cell heterogeneity would be amplified by the distribution of siRNA effect.
While we previously mentioned variability in drug uptake as a possible source of cell heterogeneity, we have expanded this section to make the reference to siRNA more explicit:

\begin{manuscript}{Pages 6-7}
It is also possible that damping rate changes due to siRNA or small molecule perturbation could be manifested through altering the system’s extrinsic noise. 
{\bfseries Such a change could be caused by an unequal uptake of siRNA or drug on a cell-to-cell basis, as has been demonstrated by a distribution of single-cell knockdown efficiency through flow cytometry (27). 
This effect would increase cell heterogeneity and lead to faster} dephasing kinetics.
\end{manuscript}

While it is also true that different siRNA constructs will have different knockdown effectivenesses, this will be a systematic effect that will not effect either the intrinsic or extrinsic noise.
In developing our approach, we do not assume that each siRNA will be equally effective - just as the knockdown of different genes will have varied effects on circadian rhythms.
While we do group siRNA perturbations by target gene, this test mainly serves to guard against off-target activities.
We have added a relevant explanation to the methods section clarifying this point:

\begin{manuscript}{Page 10}
A Hotelling's $T^2$ test was used to determine whether the means of each gene knockdown was significantly different from the control population.
{\bfseries While different siRNA constructs will have different knockdown efficiencies, grouping based on gene target helps to eliminate the effect of off-target activities.}
\end{manuscript}

\vspace{3ex}

Once more, we would like to thank each of our reviewers for their time and attention to detail.
We believe their follow-up critiques have further improved the rigor of this work, and have included our appreciation for their efforts in the acknowledgements.

\vspace{4ex}
\begin{flushright}
  Sincerely,\\[2ex]

  \includegraphics[width=0.3\textwidth]{/home/peter/Documents/Research/figures/frank_signature.pdf}\\[1ex]
Francis J. Doyle III
\end{flushright}

\end{document}

\end{document}
